\documentclass[11pt,twoside]{article} % 可选参数 shijuan, mtpro2
\usepackage{CMC}
\title{第十一届中国大学生数学竞赛预赛参考答案}
\author{14 金融工程--零蛋大}
\date{2019年10月27日}
\type{非数学类}
\examtime{150}
\watermark{48}{11}{机密}
%\renewenvironment{solution}{\setbox0\vbox\bgroup}{\egroup\unskip}
\usepackage{listings}
% settings for listings.sty
% https://www.latexstudio.net/archives/8420.html
%\usepackage[usenames, dvipsnames]{xcolor}
\definecolor{ForestGreen}{rgb}{0.0,0.34,0.0}
\definecolor{frenchplum}{RGB}{129,20,83}
\renewcommand{\lstlistingname}{代码清单}
\lstdefinestyle{lfonts}{
  basicstyle   = \footnotesize\ttfamily,
  stringstyle  = \color{purple},
  keywordstyle = \color{blue!60!black}\bfseries,
  commentstyle = \color{olive}\normalfont,
}
\lstdefinestyle{lnumbers}{
  numbers     = left,
  numberstyle = \tiny,
  numbersep   = 1em,
  firstnumber = 1,
  stepnumber  = 1,
}
\lstdefinestyle{llayout}{
  breaklines       = true,
  tabsize          = 2,
  columns          = flexible,
}
\lstdefinestyle{lgeometry}{
  xleftmargin      = 20pt,
  xrightmargin     = 0pt,
  frame            = tb,
  framesep         = \fboxsep,
  framexleftmargin = 20pt,
}
\lstdefinestyle{lgeneral}{
  style = lfonts,
  style = lnumbers,
  style = llayout,
  style = lgeometry,
}
\def\beginlstdelim#1#2#3{%
  \def\endlstdelim{#2\egroup}%
  \ttfamily#1\bgroup\color{#3}\aftergroup\endlstdelim}
\lstdefinestyle{ldelims}{
  moredelim = **[is][\beginlstdelim{\$}{\$}{orange}]{\$}{\$},
  moredelim = **[is][\beginlstdelim{\{}{\}}{ForestGreen}]{\{}{\}},
  moredelim = **[is][\beginlstdelim{[}{]}{cyan}]{[}{]},
}
% LaTeX lst style
\lstdefinestyle{lltx}{
  language = {[LaTeX]TeX},
  style = lgeneral,
  style = ldelims,
  morekeywords = {% LaTeX original commands
    maketitle,
    xingkai,kaishu,songti,heiti,fangsong,zhongsong,
    rmfamily, sffamily, ttfamily,
    itshape, slshape, scshape,
    mdseries, bfseries, emph,
    textrm, textsf, texttt,
    textit, textsl, textsc,
    textmd, textbf,bfit,
    newcommand, renewcommand, providecommand,
    cs, meta, marg, oarg, parg,
    defen,score
  }
}
\lstdefinestyle{tsdtex}{
	language = {[LaTeX]TeX},
	style = lfonts,
	style = llayout,
	%style = lgeometry,
	style = ldelims,
	breaklines=true,
	numbers=none,
	morekeywords = {% LaTeX original commands
		maketitle,
		xingkai,kaishu,songti,heiti,fangsong,zhongsong,
		rmfamily, sffamily, ttfamily,
		itshape, slshape, scshape,
		mdseries, bfseries, emph,
		textrm, textsf, texttt,
		textit, textsl, textsc,
		textmd, textbf,bfit,
		newcommand, renewcommand, providecommand,
		cs, meta, marg, oarg, parg,
		defen,score
	}
}
\lstdefinestyle{iltx}{
  style      = lltx,
  basicstyle = \ttfamily
}
\lstdefinestyle{lbash}{
  language   = {bash},
  style      = lgeneral,
}
\lstdefinestyle{ibash}{
  style      = lbash,
  basicstyle = \ttfamily
}
\definecolor{winered}{rgb}{0.5,0,0}
\definecolor{structurecolor}{RGB}{60,113,183}
\lstset{language=[LaTeX]TeX,
	texcsstyle=*\color{winered},
	numbers=none,
	breaklines=true,
	keywordstyle=\color{winered},
	commentstyle=\color{gray},
	emph={elegantpaper,fontenc,fontspec,xeCJK,FiraMono,xunicode,newtxmath,figure,fig,image,img,table,itemize,enumerate,newtxtext,newtxtt,ctex,microtype,description,times,newtx,booktabs,tabular,PDFLaTeX,XeLaTeX,type1cm,BibTeX,device,color,mode,lang,amsthm,tcolorbox,titlestyle,cite,marginnote,ctex,listings},
	emphstyle={\color{frenchplum}},
	morekeywords={DeclareSymbolFont,SetSymbolFont,toprule,midrule,bottomrule,institute,version,includegraphics,setmainfont,setsansfont,setmonofont ,setCJKmainfont,setCJKsansfont,setCJKmonofont,RequirePackage,figref,tabref,email,maketitle,keywords,definecolor,extrainfo,logo,cover,subtitle,appendix,chapter,hypersetup,mainmatter,tableofcontents,elegantpar,numbers,authoryear,heiti,kaishu,lstset,pagecolor,zhnumber,marginpar,part,equote},
	frame=single,
	tabsize=2,
	rulecolor=\color{structurecolor},
	framerule=0.2pt,
	columns=flexible,
	% backgroundcolor=\color{lightgrey}
}
\endinput

%\lstinline[style=iltx]|命令|
\begin{document}
\maketitle
\begin{flushleft}
\zihao{-4}
\begin{tabular}{|m{2.8em}<{\centering}|*{7}{m{(0.78\textwidth-2.8em)/7}<{\centering}|}}
\hline
题号 & 一 & 二 & 三   & 四 & 五 & 六  & 总~~分 \\\hline
满分 & 30 & 14 & 14  & 14 & 14  & 14 & \raisebox{0.4em}{100}\rule{0pt}{8mm}\\
\hline
得分 &    &    &     &    &    &    &\rule{0pt}{8mm} \\
\hline		
\end{tabular}\vspace*{0.6em}		
$\begin{aligned}
\mbox{注意:}
&\text{本试卷共六大题, 满分100分, 考试时间为150分钟.}\\
&1.\,\mbox{所有答题都须写在试卷密封线右边,写在其他纸上一律无效}.\\
&2.\,\mbox{密封线左边请勿答题,密封线外不得有姓名及相关标记}.\\
&3.\,\mbox{如答题空白不够,可写在当页背面,并标明题号}.
\end{aligned}$		
\end{flushleft}
\setlength{\marginparsep}{-0.8cm}
%%==================================================================
%%—————————————————————————————正文开始———————————————————————————————
%%==================================================================		
	

\section{首页的计分表格}
官方的题数无规律,只能单独设置, 作为示例
\begin{lstlisting}[style=tsdtex]
\begin{tabular}{|m{2.8em}<{\centering}|*{7}{m{(0.78\textwidth-2.8em)/7}<{\centering}|}}
\hline
题号 & 一 & 二 & 三   & 四 & 五 & 六  & 总~~分 \\\hline
满分 & 30 & 14 & 14  & 14 & 14  & 14 & \raisebox{0.4em}{100}\rule{0pt}{8mm}\\
\hline
得分 &    &    &     &    &    &    &\rule{0pt}{8mm} \\
\hline		
\end{tabular}
\end{lstlisting}

\section{cmc模板的字体说明}

\begin{enumerate}
\item 中文字体说明
\begin{itemize}
\item 主要使用的字体为思源Noto系列字体, 需要自行下载安装. 
\begin{itemize}
\item 右键, 为所有用户安装(A)
\end{itemize}
\item 自定义的字体:华文中宋、华文行楷
\begin{enumerate}
\item \lstinline[style=iltx]|{\zhongsong 华文中宋}| {\zhongsong 华文中宋}、\lstinline[style=iltx]|{\xingkai 华文行楷}| {\xingkai 华文行楷}
\item 若系统不存在此字体,则不定义
\end{enumerate}
\end{itemize}
\item 英文字体说明
\begin{enumerate}
\item 英文字体粗斜:
\begin{enumerate}
	\item 推荐 \lstinline[style=iltx]|$\bm{abc}$| \quad $\bm{abc}$
	\item 不推荐 \lstinline[style=iltx]|\bfit{Times New Roman Bold Italic}| \bfit{Times New Roman Bold Italic}
\end{enumerate}
\end{enumerate}
\item 数学字体说明
\begin{itemize}
\item 默认的数学字体:XITS Math+Cambria Math
\begin{enumerate}
\item 若系统不存在 Cambria Math 字体, 则全部使用XITS Math字体
\item 若存在Cambria Math字体, 则 \lstinline[style=iltx]|\int,\iint,\iiint,\oint,\oiint|将会被替换为Cambria Math字体
\end{enumerate}
\item 安装 mtpro2 宏包, 可选参数 math=mtpro2, 或者mtpro2. 默认不选
\begin{enumerate}
\item \href{http://www.latexstudio.net/archives/241.html}{LaTeX技巧693:安装 MathTime Professional 2 数学字体}
\end{enumerate}
\end{itemize}
\end{enumerate}

\section{填空题}
\lstinline[style=iltx]|\blank[默认1.5cm]{填空题答案}| \blank{填空题答案} 
\begin{itemize}
\item 若答案 $<1.5$cm, 则下划线长度为1.5cm. 如:\blank{$\frac{1}{2}$} 
\item 若答案长度 $\geqslant1.5$cm, 则下划线长度为答案的长度. 如:\blank{填空题答案} 
\end{itemize}
\begin{lstlisting}
%\renewenvironment{solution}{\setbox0\vbox\bgroup}{\egroup\unskip} %注释则不显示答案
\begin{solution}
solution 为填空题的解答环境. 有时候不需要保留填空题的完整过程
\end{solution}
\end{lstlisting}

\section{大题}
\lstinline[style = iltx]|\section{大题}|和
\lstinline[style=iltx]|\dati[可选缩进长度]{大题}|
\begin{itemize}
\item \lstinline[style=iltx]|\dati[可选缩进长度]{大题}| 的可缩进长度目前为编号前的缩进长度. 
\end{itemize}
\begin{lstlisting}[style=tsdtex]
\dati{(本题15分) 在空间直角坐标系中,设单叶双曲面 $\Gamma$ 的方程为 $x^2+y^2-z^2=1$,设 $P$ 为空间的平面, 它交 $\Gamma$ 于一抛物线 $C$. 求该平面的法线与 $z$- 轴的夹角}
\end{lstlisting}
\dati{(本题15分) 在空间直角坐标系中,设单叶双曲面 $\Gamma$ 的方程为 $x^2+y^2-z^2=1$,设 $P$ 为空间的平面, 它交 $\Gamma$ 于一抛物线 $C$. 求该平面的法线与 $z$- 轴的夹角}\vspace{1ex}

\vspace{2.5ex}
\noindent 解答环境: Solution, Proof, proof
%\begin{lstlisting}[style=tsdtex]
%\begin{Solution}
%Solution 环境
%\end{Solution}
%\end{lstlisting}
\begin{Solution}
Solution 环境
\end{Solution}

%\begin{lstlisting}[style=tsdtex]
%\begin{proof}
%proof 环境
%\end{proof}
%\end{lstlisting}
\begin{proof}
proof 环境
\end{proof}

%\begin{lstlisting}[style=tsdtex]
%\begin{Proof}
%Proof 环境
%\end{Proof}
%\end{lstlisting}
\begin{Proof}
Proof 环境
\end{Proof}

\vspace{1.5em}
\noindent 分数线的设置
\begin{itemize}
\item 推荐 \lstinline[style=iltx]|\defen[默认0cm]{得分}|、\lstinline[style=iltx]|\score[默认0cm]{得分}|
\item 过时命令 \lstinline[style=iltx]|\dfsxian{得分}|,\lstinline[style=iltx]|\cfsxian{得分}|,\lstinline[style=iltx]|\fsxian{长度(cm)}{得分}|
\end{itemize}



\section{选择题}

\begin{lstlisting}[style=tsdtex]
\begin{tasks}(2) % 1,2,4
\task $F'(x)G'(x)=1$
\task $f'(x)g'\big(f(x)\big)=1$
\task $\frac{\dif G\big(f(x)\big)}{\dif x}=-1$
\task $\frac{\dif F\big(g(x)\big)}{\dif x}=1$	
\end{tasks}
\end{lstlisting}
\begin{tasks}(2) % 1,2,4
\task $F'(x)G'(x)=1$
\task $f'(x)g'\big(f(x)\big)=1$
\task $\frac{\dif G\big(f(x)\big)}{\dif x}=-1$
\task $\frac{\dif F\big(g(x)\big)}{\dif x}=1$	
\end{tasks}



%%%%%%%%%%%%%%%%%%%%%%%%%%%%%%%%%%%%%%%%%%%%%%%%%%%%%%%%%%%%%%%%%%%%%
%-------------------------------结束---------------------------------
%%%%%%%%%%%%%%%%%%%%%%%%%%%%%%%%%%%%%%%%%%%%%%%%%%%%%%%%%%%%%%%%%%%%%
\clearpage
\end{document}
\section{一面双页}

先保存再编译,1-6 是从第一页到第六页
\begin{verbatim}
\documentclass{article}
\usepackage{pdfpages}
\usepackage[paperwidth=40cm,paperheight=27.5cm]{geometry}
\begin{document}
\includepdf[pages=1-6,nup=2x1,scale=1,
offset=3mm 0mm,column,delta=-10 -0mm]{17mathSJ.pdf}	
\end{document}
\end{verbatim}
