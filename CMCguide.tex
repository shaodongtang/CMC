\documentclass[11pt,twoside,space]{article}
\usepackage{CMC}
\title{第十届全国大学生数学竞赛决赛参考答案}
\author{14 金融工程--零蛋大}
\date{2019年3月30日}
\type{非数学类}
\examtime{150}
\watermark{48}{11}{机密}
\begin{document}
\maketitle%\vspace*{-0.8em}
\begin{center}
	\zihao{-4}
	\begin{tabular}{|m{2.6em}<{\centering}|*{8}{m{3em}<{\centering}|}}
		\hline
		题~号 & 一 & 二 & 三   & 四 & 五 & 六  & 七  &总~~分 \\\hline
		满~分 & 30 & 12  & 12  & 12 & 12 & 11  & 11  &\raisebox{0.4em}{100}\rule{0pt}{8mm}\\
		\hline
		得~分 &    &    &     &    &    &     &    &\rule{0pt}{8mm} \\
		\hline	
	\end{tabular}\vspace*{0.6em}		
	$\begin{aligned}
	\mbox{注意:}
	&\text{本试卷共七大题, 满分100分, 考试时间为150分钟.}\\
	&1.\,\mbox{所有答题都须写在试卷密封线右边,写在其他纸上一律无效}.\hspace{12.0cm}\\
	&2.\,\mbox{密封线左边请勿答题,密封线外不得有姓名及相关标记}.\\
	&3.\,\mbox{如答题空白不够,可写在当页背面,并标明题号}.\\[-2mm]
	\end{aligned}$	
\end{center}
\setlength{\marginparsep}{-0.8cm}
%%==================================================================
%%—————————————————————————————正文开始———————————————————————————————
%%==================================================================	
	

\section{表格}
官方的题数无规律,只能单独设置

\section{中文字体说明}

\begin{table}[htbp]
\caption{中文字体说明}
\centering
\begin{tabular}{|cll|}
\hline 
\songti adobe宋体    & Adobe Song Std L     & \verb|\songti adobe宋体|    \\
\kaishu adobe楷体    & Adobe Kaiti Std R    & \verb|\kaishu adobe楷体|    \\
\heiti adobe黑体     & Adobe Heiti Std R    & \verb|\heiti adobe黑体|     \\
\fangsong adobe仿宋  & Adobe Fangsong Std R & \verb|\fangsong adobe仿宋|  \\
\zhongsong 华文中宋  & STZhongsong          & \verb|\zhongsong 华文中宋|   \\
\xingkai 华文行楷    & STXingkai            &  \verb|\xingkai 华文行楷|    \\
\hline	
\end{tabular} 
\end{table}

\section{英文字体说明}

英文字体粗斜:
\begin{enumerate}
\item 推荐使用 \verb|$\bm{abc}$|$\bm{abc}$
\item 不推荐 \verb|\bfit{Times New Roman Bold Italic}| \bfit{Times New Roman Bold Italic}
\end{enumerate}

引入数学字体mtpro2宏包,mtpro2宏包的安装参考 LaTeX技巧693:安装 MathTime Professional 2 数学字体\url{http://www.latexstudio.net/archives/241.html}

弃Times New Roman 风格

\section{解答环境}
Proof,Solution
\begin{enumerate}
\item Solution 环境
\begin{Solution}
	9899 Solution 环境
\end{Solution}
\item Proof 环境
\end{enumerate}


\verb|\dfsxian{得分}|,\verb|\cfsxian{得分}|,\verb|\fsxian{长度(cm)}{得分}|

\begin{enumerate}
\item 短分数线示例 \verb|\dfsxian{9}|
      \dfsxian{9}
\item 长分数线示例 \verb|\dfsxian{10}|
      \cfsxian{10}    
\item 任意长度分数线示例(1cm) \verb|\fsxian{1}{10}|
\fsxian{1}{10}
\end{enumerate}


\section{大题}
\verb|\section{大题}|, \verb|\dati{}{大题}| 二选一\par
\dati{}{(本题15分)\;\;%\\[2mm]
	在空间直角坐标系中,设单叶双曲面 $\Gamma$ 的方程为 $x^2+y^2-z^2=1$,设 $P$ 为空间的平面, 它交 $\Gamma$ 于一抛物线 $C$. 求该平面的法线与 $z$- 轴的夹角.\\} 

\section{选择题}
设可导函数 $f(x)$ 的原函数是 $F(x)$, 可导函数 $g(x)$ 的原函数是 $G(x)$, $g(x)$ 是 $f(x)$ 在区间 $I$ 上的反函数,  则\hfill(\qquad)\\[-1.2em]
\begin{tasks}(2) % 1,2,4
\task $F'(x)G'(x)=1$
\task $f'(x)g'\big(f(x)\big)=1$
\task $\frac{\dif G\big(f(x)\big)}{\dif x}=-1$
\task $\frac{\dif F\big(g(x)\big)}{\dif x}=1$	
\end{tasks}

\section{一面双页}

先保存再编译,1-6 是从第一页到第六页
\begin{verbatim*}
\documentclass{article}
\usepackage{pdfpages}
\usepackage[paperwidth=40cm,paperheight=27.5cm]{geometry}
\begin{document}
\includepdf[pages=1-6,nup=2x1,scale=1,
offset=3mm 0mm,column,delta=-10 -0mm]{17mathSJ.pdf}	
\end{document}
\end{verbatim*}



%%%%%%%%%%%%%%%%%%%%%%%%%%%%%%%%%%%%%%%%%%%%%%%%%%%%%%%%%%%%%%%%%%%%%
%-------------------------------结束---------------------------------
%%%%%%%%%%%%%%%%%%%%%%%%%%%%%%%%%%%%%%%%%%%%%%%%%%%%%%%%%%%%%%%%%%%%%
\clearpage
\end{document}
